% \smalltitle{سوال 1.1}
% \begin{enumerate}
%     \item \phantom{text}
%     \\
%     \begin{latin}
%         \noindent
%         $S \rightarrow aSd|X|Y|Z$\\
%         $X \rightarrow aXc|Y$\\
%         $Y \rightarrow bYc|\epsilon$\\
%         $Z \rightarrow bZd|Y$
%     \end{latin}
%     \item \phantom{text}
%     \\
%     \begin{latin}
%         $S \rightarrow ab | Xb$\\
%         $X \rightarrow a|Y$\\
%         $Y \rightarrow aXb | bXa | aXa | bXb$
%     \end{latin}
% \end{enumerate}
برای حل این سوال فرض می‌کنیم تسک‌های ما همگی زمان‌بندپذیر
\footnote{feasable}
طبق الگوریتم 
EDF
هستند و فرض کنید یک زمانبدی غیر 
EDF
داریم و بدون کم شدن از کلیات مسئله فرض می‌کنیم زمانبند ما نیز برای تسک‌های تعریف‌شده 
feasable
است.
\\
حال فرض کنید طبق زمانبد ما تعدادی تسک به صورت زیر زمانبندی شده‌اند به این صورت که می‌دانیم قطعا این زمانبند feasable 
است اما ددلاین $T_1$
دورتر از ددلاین $T_2$ است.
\\
\begin{latin}
  Feasable non-EDF scheduling : 
\tikzset{
  gray box/.style={
    fill=gray!20,
    draw=gray,
    minimum width={2*#1ex},
    minimum height={2em},
  },
  annotation/.style={
    anchor=north,
  }
}
\begin{tikzpicture}[node distance=-0.5pt]
  \node [gray box=7] (p1) {\(T_{1}\)};
  \node [gray box=3, right=of p1] (p2) {\(\hdots\)};
  \node [gray box=6, right=of p2] (p3) {\(T_{2}\)};
  % \node [gray box=3, right=of p3] (p4) {\(P_{4}\)};

  \node [annotation] at (p1.south west) {$r_1$};
  \node [annotation] at (p1.south east) {$r_1 + c_1$};
  \node [annotation] at (p2.south east) {$r_2$};
  \node [annotation] at (p3.south east) {$r_2 + c_2$};
  % \node [annotation] at (p4.south east) {30};
\end{tikzpicture}
\\
\end{latin}
حال فرض می‌کنیم جای دو تسک $T_1$ و $T_2$ عوض شود.
\begin{latin}
  EDF scheduling : 
\tikzset{
  gray box/.style={
    fill=gray!20,
    draw=gray,
    minimum width={2*#1ex},
    minimum height={2em},
  },
  annotation/.style={
    anchor=north,
  }
}
\begin{tikzpicture}[node distance=-0.5pt]
  \node [gray box=6] (p1) {\(T_{2}\)};
  \node [gray box=3, right=of p1] (p2) {\(\hdots\)};
  \node [gray box=7, right=of p2] (p3) {\(T_{1}\)};
  % \node [gray box=3, right=of p3] (p4) {\(P_{4}\)};

  \node [annotation] at (p1.south west) {$r_1$};
  \node [annotation] at (p1.south east) {$r_1 + c_2$};
  \node [annotation] at (p2.south east) {$r'_1$};
  \node [annotation] at (p3.south east) {$r'_1 + c_1$};
  % \node [annotation] at (p4.south east) {30};
\end{tikzpicture}
\\
\end{latin}
حال می‌دانیم چون زمانبند اول ما feasable است و $\leftarrow D_{T_2} < D_{T_1} $ 
\begin{latin}
  \noindent
  $
    r_1 + c_1 \leq D_{T_1} \\ r_2 + c_2 \leq D_{T_2} < D_{T_1}\\ 
    r_1 + c_1 < r_2 \\
    r_1 + c_2 < r'_1
  $
\end{latin}
همچنین چون فقط جای ۲ تسک را عوض کرده‌ایم، پس زمان کلی ما ثابت مانده است $\leftarrow$
\begin{latin}
  \noindent
  $
  r_2 + c_2 = r'_1 + c_1  \\ 
  r'_1 + c_1 = r_2 + c_2 \leq D_{T_2} < D_{T_1} \Rightarrow r'_1 + c_1 < D_{T_1} \Rightarrow \text{Stays feasable for } T_1 \checkmark \\
  r_1 + c_2 < (r'_1+c_1=r_2+c_2) \leq D_{T_2} \Rightarrow r_1 + c_2 < D_{T_2} \Rightarrow \text{Stays feasable for } T_2 \checkmark \\
  $
\end{latin}

حال اثبات کردیم که با تعویض این دو تسک باز هم زمانبند ما feasable 
باقی می‌ماند پس EDF از نظر 
feasability 
حالت optimal برای ما است.
% با معادلات بالا نیز می‌توان اثبات کرد 