\smalltitle{سوال 2.1}
\begin{enumerate}
    \item \phantom{text}
          \\
          فرض کنید L1 زبان تولید شده توسط گرامر G1=(V1,T,P1,S1) باشد و L2 زبان تولید شده توسط گرامر G2=(V2,T,P2,S2) باشد. حال می‌خواهیم
          یک گرامر جدید G=(V,T,P,S) تولید کنیم که L1L2 را تشخیص دهد.
          \\
          \begin{latin}
              V = V1 U V2 U {S}\\
              T = T\\
              P = P1 U P2 U {S $\rightarrow$ S1S2}\\
              S = S
          \end{latin}
          حالا پس توانستیم با کمک ۲ گرامر،‌ یک گرامر جدید تولید کنیم که می‌تواند الحاق را تشخیص دهد پس عملا زبان‌های CFL تحت الحاق بسته هستند.

    \item \phantom{text}
    \\
    فرض کنید G گرامری باشد که زبان L را تولید می‌کند. حال می‌خواهیم گرامر جدید G' تولید کنیم.
    \begin{latin}
        G = (V,$\sum$,R,S)\\
        G'= (V',$\sum$,R',S')\\
        V' = V U {S'}\\
        R' = \{S' $\rightarrow$ S | $\epsilon$, $\forall A \rightarrow a \in R : \text{add }A \rightarrow aS'$\}
    \end{latin}
    حال قوانین نشان می‌دهند که می‌توان اسپیولن ترزیشن داشت و طبق استقرا می‌توان اثبات کرد که دومین قانون به ما اجازه می‌دهد رشته‌های تولید شده توسط گرامر اول را الحاق کنیم ( که از قسمت قبلی می‌دانیم بسته است)
\end{enumerate}