\smalltitle{سوال 4.2}
\begin{enumerate}
    \item \phantom{text}
          \\
          ابتدا فرض می‌کنیم زبان ما مستقل از متن است و رشته دلخواه $s=a^{2p}b^{3p}c^{p}$ را در آن درنظر می‌گیریم. حال برای رشته vxy حالت‌بندی می‌کنیم.
          \\حالت اول این است که vxy درون یکی از رشته‌های $a^{2p},b^{3p},c^{p}$ باشد. می‌دانیم که در این صورت پس از اتمام لم پمپاژ طول یکی از a,b,c زیاد می‌شود پس توان‌های آنها از رابطه گفته‌شده در زبان تخطی کرده و دیگر در زبان نیستند.
          \\
          حالت دوم این است که $vxy \in aa\cdots ab\cdots b $ که در این صورت چون حتما پس از انجام عملیات پمپاژ طول حداقل یکی از رشته‌های a یا b زیاد می‌شود
          پس می‌دانیم قطعا توان عبارتی که زیاد شده از ۲ یا ۳ برابر توان c بزرگ‌تر است (چون c اضافه نشده اما حداقل یکی از a یا b طول بیشتری پیدا کرده) پس در این حالت نیز رشته جدید در زبان ما نیست.\\
          حالت آخر هم این است که $vxy \in b\cdots b c\cdots c$ که در این صورت پس از پمپاژ طول حداقل یکی از رشته‌های b یا c زیاد می‌شود پس چون طول رشته a در این فرایند تغییری نکرده پس رابطه بین طول رشته زیاد شده با a بر هم میخورد و می‌توان مشاهده کرد رشته جدید عضو $L_1$ نیست.
          \\
          پس تمامی حالات را بررسی کردیم و نشان دادیم در هیچ حالتی با پمپاژ کردن، رشته جدید در زبان باقی نمی‌ماند پس فرض ما رد می‌شود و زبان ما مستقل از متن نبوده است.

    \item \phantom{text}
          \\
          برای این قسمت رشته $s=a^{(p+1-1)(p+1+1)}$ را در نظر می‌گیریم.
          \\
          حال این رشته به صورت زیر در می‌آید.
          \\
          \begin{latin}
            $s=a^{(p)(p+2)} \xrightarrow[]{\text{after pumping}}s'=a^{p^2+2p+x}$
          \end{latin}
          حال توان عبارت جدید را در نظر بگیرید. x ما عملا هر عددی می‌تواند باشد پس آنرا 1 در نظر می‌گیریم. در این صورت توان ما به شکل $p^2+2p+1 = (p+1)^2$ در می‌آید.
          حال برای این که این این رشته در زبان باشد، باید n وجود داشته باشد به صورتی که $n^2-1=(p+1)^2\xrightarrow[]{p+1=x}n^2-1=x^2$
          می‌توان اثبات کرد که همچنین عددی وجود ندارد به این صورت که چون تابع $x^2$ در بازه اعداد طبیعی اکیدا صعودی است پس بررسی آن در نقاط 1 و 2 کافی است چرا که اعداد از ۰ بزرگتر هستند و پس از این نقاط نیز اختلاف دو عدد مربع کامل قطعا از ۱ بیشتر می‌شود (به دلیل شیب زیاد).\\
          \begin{latin}
            if n = 2, x=1 $\rightarrow$ 4 = 3 $\rightarrow$ contradiction
          \end{latin}
          حال اثبات کردیم همچین عددی نمی‌تواند در لیست اعداد مجذور منهای ۱ باشد 
          پس رشته ساخته شده در زبان نیست و فرض مستقل از متن بودن زبان غلط است.

          \item \phantom{text}
          \\
          برای حل این سوال رشته $s=a^pb^pc^i$ را در نظر می‌گیریم. طبق حالت‌بندی یا vxy در a است یا در b است و یا بین این دو و یا در c و یا بین b , c.\\
          \\
          اگر $vxy \in a^p$ : 
          در این صورت پس از پمپاژ طول a افزایش می‌یابد پس دیگر با b برابر نیست و رشته جدید در زبان ما نیست.
          \\
          اگر $vxy \in b^p$ : 
          در این صورت پس از پمپاژ طول رشته b افزایش یافته پس برابر a نیست و رشته جدید در زبان نیست.\\
          اگر $vxy \in c^i$ :
          می‌توان به قدری پمپاژ انجام داد که توان جدید c بزرگ‌تر از 2p بشود پس رشته جدید در زبان نیست.\\
          اگر $vxy \in b^nc^i$ :
          یا طول b افزایش می‌یابد که در این صورت چون a ثابت مانده پس طول جدید این دو یکی نیست و یا c افزایش می‌یابد که مطابق قسمت قبل می‌توان کاری کرد از 2p بیشتر شود و یا هر دو زیاد می‌شوند که طول b با a برابر نمی‌ماند.
          \\
          اگر $vxy \in a^pb^p$ : 
         در این صورت چون طول رشته vxy حداکثر برابر p است پس اگر فرض کنیم هر دوی a و b به یک اندازه زیاد نشوند پس طول آنها نهایتا فرق می‌کند و رشته جدید در زبان نیست و اگر به یک اندازه زیاد شوند مثلا x می‌توان x را به حدی زیاد کرد که دیگر رابطه i بین p+x و 2p+2x صدق نکند که یعنی رشته جدید باز هم در زبان نیست.\\
         پس اثبات کردیم در هیچ حالتی رشته جدید نمی‌تواند در زبان باشد پس زبان مستقل از متن نیست.

        
\end{enumerate}